%% polematrix documentation
%% Copyright (C) 2017 Jan Felix Schmidt <janschmidt@mailbox.org>
%%   
%% This program is free software: you can redistribute it and/or modify
%% it under the terms of the GNU General Public License as published by
%% the Free Software Foundation, either version 3 of the License, or
%% (at your option) any later version.
%%
%% This program is distributed in the hope that it will be useful,
%% but WITHOUT ANY WARRANTY; without even the implied warranty of
%% MERCHANTABILITY or FITNESS FOR A PARTICULAR PURPOSE.  See the
%% GNU General Public License for more details.
%%
%% You should have received a copy of the GNU General Public License
%% along with this program.  If not, see <http://www.gnu.org/licenses/>.

\documentclass[a4paper]{scrartcl}
\usepackage[english]{babel}
\usepackage[utf8]{inputenc}
\usepackage[T1]{fontenc}
\usepackage[margin=2.5cm]{geometry}
\usepackage{siunitx}
\usepackage{csquotes}
\usepackage{booktabs}
\usepackage{multicol}
\usepackage{amsmath}
\usepackage[colorlinks,linkcolor=blue,citecolor=magenta,urlcolor=teal]{hyperref}
\usepackage[backend=biber,style=alphabetic,backref]{biblatex}
\usepackage{todonotes}

\usepackage{polem_defs}
\usepackage{cleveref}

\author{Jan Felix Schmidt \textless janschmidt@mailbox.org\textgreater}
\title{polematrix}
\subtitle{a spin tracking code for electron accelerators using elegant}
\date{documentation for version 1.0}

\hypersetup{
  pdftitle={polematrix documentation},
  pdfauthor={Jan Felix Schmidt}
}

\bibliography{polem}

\setcounter{tocdepth}{2}

\begin{document}
\maketitle

\begin{abstract}
  \polem is a spin tracking code. It calculates the motion of the spin of ultra
  relativistic electrons in particle accelerators. It implements a simple matrix tracking
  as multi-thread \cpp application. The input can be any lattice file from \ele or
  \madx. All required particle trajectories are also imported from the established
  particle tracking codes \ele or \madx.

  \polem includes synchrotron radiation via import of particle trajectories from \ele or
  via an own implementation of longitudinal phasespace. A short description of \polem is
  given in the IPAC'16 contribution \cite{IPAC16-decoh} and in my phd thesis \cite{dr},
  which will be available soon (in German).
  
  Do not hesitate to contact me if you have any questions and please report bugs.
\end{abstract}

\tableofcontents
\clearpage


\section{\polem can}
\label{sec:polem-can}

\begin{itemize}
\item perform element by element electron spin tracking
\item read a lattice from common \ele or \madx files (see \cref{sec:types} for supported elements)
\item import closed orbit, optics parameters and particle trajectories from \ele or
  \madx
\item configure and execute \ele (or \madx) automatically
\item track multiple spins in parallel (any number of threads)

\item consider synchrotron radiation for beam dynamics:
  \begin{itemize}
  \item longitudinal and transversal phase space via import from \ele (or \madx)
  \item longitudinal phase space via fast own implementation
  \end{itemize}
\item simulate linear energy ramps
\item configure any magnet's field to oscillate harmonically (including linear frequency sweep)
\item simulate depolarizing resonances including synchrotron sidebands
\item estimate resonance strengths of depolarizing resonances
\item be used and modified for free under terms of the GNU General Public License
\end{itemize}
A detailed description and discussion of \polem as well as simulation examples can be
found in \cite{dr}.

\section{\polem cannot}
\label{sec:polem-cannot}

\begin{itemize}
\item simulate radiative polarization build-up (\textsc{Sokolov-Ternov} effect)
\item track spins of other particles than electrons
\item consider electric fields for spin tracking
\item perform transversal particle tracking independent of \ele and \madx
\item simulate nonlinear energy ramps
\item consider a field strength which varies longitudinally within a magnet
\end{itemize}
Some of these features could be added with little effort. Please contact me if you have
any questions.






\section{Installation}
\label{sec:installation}

\polem has been tested only on Ubuntu/Debian Linux with the GCC compiler so far.

\subsection{Dependencies}
\label{sec:dependencies}

\polem requires the following programs and libraries:
\begin{multicols}{2}
\begin{itemize}
\item GCC compiler
\item CMake
\item Gnu Scientific Library (GSL) \cite{gsl}
\item Boost program options
\item Boost filesystem
\item Boost property tree %\cite{boost-pt}
\item Boost random %\cite{boost-random}
\item Armadillo library \cite{arma}
\item \pal library \cite{palattice}
\end{itemize}
\end{multicols}
\pal is available on github. The source code can be downloaded at \cite{palattice} and
compiled and installed as described in the enclosed README file.
%
All other dependencies may be in the package repositories of your Linux distribution. On
Ubuntu you can install all required packages with the following command:
\begin{bashcode}
  sudo apt-get install g++ cmake libgsl-dev libboost-dev libboost-program-options-dev libboost-filesystem-dev libarmadillo-dev
\end{bashcode}
Additionally, \polem relies on the established particle
tracking program \ele (or \madx), whose lattice files and tracking results are used for
spin tracking.  \ele (and \madx) can be configured and execute automatically by \polem using
\pal. The installation of \ele and \madx is described in \cref{sec:elemadx}.


    
\subsection{Build and Install \polem}
\label{sec:build}
\polem is compiled and installed using \software{CMake} with the following commands:
\begin{bashcode}
  cd polematrix/build
  cmake ../
  make
  sudo make install
\end{bashcode}
The default install path is \bashinline{/usr/local/bin}. It can be changed by setting the
\bashinline{CMAKE_INSTALL_PREFIX} variable in \bashinline{polematrix/CMakeLists.txt}.
%
To uninstall \polem run
\begin{bashcode}
  cd polematrix/build
  sudo make uninstall
\end{bashcode}



\subsection{Underlying Particle Tracking Programs}
\label{sec:elemadx}

I recommend to use \polem with \ele since much more tracking parameters can be set
automatically -- including number of particles (starting with a Gaussian profile in
radiation equilibrium), tracking duration and linear energy ramps. If you have only a
lattice file in \madx format, you can convert it automatically to an \ele lattice
using the program \software{convertlattice} which is included in \pal \cite{palattice}.

\subsubsection{\ele}
\label{sec:ele}
The particle tracking program \ele \cite{elegant} is developed at the Advanced Photon
Source at Argonne National Laboratory. Packages for many operating systems can be
downloaded at \cite{elegant-download}. Also install the \software{SDDSToolKit}, which is
needed to access the binary output files of \ele.
%
If there is no package for your system, you can download the \bashinline{Build-AOP-RPMs}
script instead which automatically builds all desired programs from source (see the guide
in \cref{sec:ele-install}).
%
There is a parallel version of \ele, called \software{Pelegant}, which speeds up the
tracking \cite{pelegant} (installation hints in \cref{sec:ele-install}). The \ele manual
can be found at \cite{elegant-manual}.

\subsubsection{\madx}
\label{sec:madx}
The particle tracking program \madx is developed at CERN and can be downloaded from the
website \cite{madx}, which also provides the \madx manual.




\section{Usage}
\label{sec:usage}

\subsection{Execution}
\label{sec:execution}

\polem is called via
\begin{bashcode}
  polematrix [options] [CONFIGURATION FILE]
\end{bashcode}
where \bashinline{[options]} are the command line options shown in
\cref{tab:polem-options}. All parameters of the spin tracking can be set up in an \xml
configuration file (\bashinline{[CONFIGURATION FILE]}) which is described in
\cref{sec:config}.
%
The lattice of the accelerator is given as any conventional \ele or \madx lattice file and
is set in the configuration file (see \cref{sec:config}). A template configuration file
with default values can be generated anytime by executing \bashinline{polematrix --template}.


\begin{table}[h]
  \centering
  \begin{tabular}{ll}
    \toprule
    \bashinline{-h [ --help ]}              &  display this help message \\
    \bashinline{-V [ --version ]}           &  display version \\
    \bashinline{-T [ --template ]}          &  create config file template (\bashinline{template.pole}) and quit \\
    \bashinline{-R [ --resonance-strengths ]} &  estimate resonance strengths instead of spin tracking \\
    \midrule
    \bashinline{-t [ --threads ] arg (=all)}     &  set number of threads used for tracking \\
    \bashinline{-o [ --output-path ] arg (=.)}   &  path for output files \\
    \bashinline{-v [ --verbose ]}            &  activate additional status output \\
    \bashinline{-n [ --no-progressbar ]}     &  do not show progress bar during tracking\\
                                            &  (e.g. if output is redirected to a log file)\\
    \bashinline{-a [ --all ]}                &  write additional output files (e.g. lattice und orbit) \\
    \bashinline{-s [ --spintune ] arg}       &  in resonance strengths mode (\bashinline{-R}):\\
                                            &  calculate for given spin tune only\\
    \bottomrule
  \end{tabular}
  \caption{Command line options of \polem. They consist of special program modes in the
    upper part and configuration options in the lower part.}
  \label{tab:polem-options}
\end{table}

\subsection{Output}
\label{sec:output}

The components of each spin vector $\svec[i](t)$ is written to the text file
\bashinline{spins/spin_i.dat} in the output path during tracking. The output step width
can be set in the configuration file (see \cref{sec:config-spintrk}). When tracking is
completed for all spins, the polarization vector $\pvec(t)$ is calculated for each output
step as average over all successfully tracked spins. It is saved as
\bashinline{polarization.dat} in the output path.



\section{Coordinate System and Polarization}
\label{sec:coord}

\begin{figure}
  \centering
  \includegraphics{coord}
  \caption{Accelerator coordinate system and precessing spin vectors. \cite{dr}}
  \label{fig:coord}
\end{figure}

% Coordinate System
\polem uses the typical accelerator coordinate system sketched in \cref{fig:coord}, which
is moving along the reference orbit. The position along the reference orbit is called $s$.
Since \polem assumes ultra relativistic electrons, the velocity along $s$ is assumed to be
constant $v\approx c$ and the position is alternatively parametrized by the time
$t = s/c$. $t$ is used as parameter for multiple turns in a circular accelerator and to
define the duration and output step width of the spin tracking. The output can be fixed to
the position of a certain element in the accelerator using the entry
\xmlinline{<outElement>} in the configuration file (see \cref{sec:config}).

At each position $s$ the $x$ axis points in horizontal direction and the $z$ axis points
in vertical direction. $z>0$ is above the reference orbit. In a circular electron
accelerator $x>0$ corresponds to the outer side of the ring (to larger radii of the
bending magnets). The longitudinal axis is also referred to as $s$ axis.

% Spin and Polarization Vectors
Accordingly, the tracked spin vector $\svec[i](t)$ of electron $i$ is parametrized by a horizontal
component \sx, vertical component \sz and longitudinal component \slong. Since the spin
naturally \enquote{moves} with the corresponding electron, $\svec[i](t)$ only
includes the spin orientation at the current spacial position of the electron. The
absolute value is always $|\svec[i]| = 1$.
%
The same coordinate system is used for the resulting polarization vector
\begin{equation*}
\pvec(t) = \frac{1}{N} \sum_{i=0}^{N-1} \svec[i](t) \quad.
\end{equation*}
Its absolute value $0\leq\pabs\leq 1$ is the polarization degree. The projections \px, \pz
and \plong can be interpreted as probabilities to measure spins oriented in parallel to
the corresponding axis.
%
The $N$ tracked spins are labeled with $i\in \{0,1,\dots,N-1\}$.




\section{Brief Introduction to the Concept of \polem}
\label{sec:brief-intr}

\subsection{Spin Tracking and Magnetic Fields}
\label{sec:concept-spin-tracking}
Spin precession of a relativistic electron in a particle accelerator can be described in
the laboratory frame by the
\textsc{Thomas-BMT} equation \cite{thomas,bmt}
\begin{align}
  \nonumber
  \frac{\dif \svec}{\dif t} &= \wtbmtvec\times\svec\\
  \label{eq:tbmt}
  \text{with}\quad
  \wtbmtvec &:= - \frac{e}{\gamma m} \left[ (1+\ga)\btrans + (1+a)\bpara
    - \left(\frac{\gamma}{1+\gamma} +\ga\right)\vec\beta\times\frac{\vec E}{c} \right] \quad.
\end{align}
Here, $e$, $m$ and $a=(g_s-2)/2$ are charge, rest mass and gyromagnetic anomaly of an
electron. 
$\gamma$ is the \textsc{Lorentz} factor and \bvec and $\vec E$ are the magnetic and
electric  field at the position of the electron, where the components of \bvec have to be
distinguished with respect to the direction of motion of the electron.
%
Neglecting transversal electric fields and normalizing the magnetic fields to the magnetic
rigidity as
\begin{equation}
\label{eq:Bn}
  \bnvec := \frac{e}{\pcentral} \bvec \quad,
\end{equation}
the \textsc{Thomas-BMT} equation can be simplified to
\begin{equation}
  \label{eq:tbmt-short}
  \frac{\dif \svec}{\dif t} \approx - c\cdot \left[ (1+\ga)\bntrans + (1+a)\bnpara \right]
  \times\svec \quad.
\end{equation}
%
This equation describes a precession of \svec around any magnetic field vector. The precession
frequency around a vertical field \bdip of a bending magnet is
\begin{equation}
  \label{eq:omegaDip}
  \Omega_{\text{TBMT}, z} = - \frac{e}{\gamma m} (1+\ga) \bdip = (1+\ga) \wrev
\end{equation}
with the revolution frequency in a circular accelerator
\begin{equation*}
  \wrev = - \frac{e}{\gamma m} \bdip \quad.
\end{equation*}


Now the revolution frequency \wrev is subtracted from the precession frequency to
transform \cref{eq:tbmt-short} from the laboratory frame to the co-moving accelerator
coordinate system introduced in \cref{sec:coord}, which rotates in the lab frame once per
revolution (as the particle's momentum). This yields the equation
\begin{equation}
  \label{eq:tbmt-pole}
  \frac{\dif \svec(t)}{\dif t} \approx - c\cdot \left[ \gamma(t)a\bntrans(t) + a\bnpara(t) \right]
  \times\svec(t) \quad,
\end{equation}
where all time dependencies are now given explicitly.
%
Both $\gamma(t)$ and $\bvec(t)$ depend on the time $t$:
\begin{itemize}
\item In case of the magnetic field $\bvec(t)$, $t$ parameterizes two dependencies. First, the field
  depends on the position $s$ in the lattice; second, for some element types it depends on the transversal position
  $(x,z)$ of the electron relative to the reference orbit. The
  element positions and parameters like bending radius $R$, quadrupole strength $k_1$ and
  sextupole strength $k_2$ are imported from the \ele or \madx lattice file. The formula
  for field calculation is given in \cref{sec:concept-b}. The transversal
  trajectories $x(t)$ and $z(t)$ can be considered by different models described in
  \cref{sec:concept-traj}.
  \item The \textsc{Lorentz} factor $\gamma(t)$ is proportional to the particle's energy and
  therefore depends on its longitudinal dynamics, which can be considered by different
  models described in \cref{sec:concept-gamma}.
\end{itemize}

\Cref{eq:tbmt-pole} describes a three dimensional rotation of \svec around \bnvec with variable
speed. \polem models this spin motion by three dimensional rotation matrices using the
following assumptions/approximations:
\begin{itemize}
\item The spin does not precess in-between the magnets because of $\bnvec=0$ in the drift
  spaces.
\item The magnetic field of an element does not change along the particles path within the
  element.
\item The particle's angle to the reference orbit can be neglected and \bpara can be set
  to \blong.
\end{itemize}
Hence, the spin rotation can be tracked element by element. This means, the spin
precession from a position $s_1$ in front of an element with reference position $s_e$ to a
position $s_2$ behind the element is calculated as
\begin{equation}
  \label{eq:matrixtracking}
  \svec(s_2) = \mathbf{R}_{\hat{r}}(\sang) \cdot\svec(s_1)
\end{equation}
with the general three dimensional rotation matrix
\begin{equation*}
  \mathbf{R}_{\hat{r}}(\sang) = 
  \begin{pmatrix}
    \hat{r}_x^2\left(1-\cos \sang \right)+\cos \sang &\hat{r}_x\hat{r}_s\left(1-\cos \sang
    \right)-\hat{r}_z\sin \sang &\hat{r}_x\hat{r}_z\left(1-\cos \sang \right)+\hat{r}_s\sin
    \sang \\
    \hat{r}_s\hat{r}_x\left(1-\cos \sang \right)+\hat{r}_z\sin \sang &\hat{r}_s^2\left(1-\cos \sang
    \right)+\cos \sang &\hat{r}_s\hat{r}_z\left(1-\cos \sang \right)-\hat{r}_x\sin \sang \\
    \hat{r}_z\hat{r}_x\left(1-\cos \sang \right)-\hat{r}_s\sin \sang &\hat{r}_z\hat{r}_s\left(1-\cos \sang
    \right)+\hat{r}_x \sin \sang &\hat{r}_z^{2}\left(1-\cos \sang \right)+\cos \sang
  \end{pmatrix}
\end{equation*}
for a rotation with the angle \sang around the axis $\hat{r}$. Rotation angle and axis are
given by \cref{eq:tbmt-pole} and the integrated field of the element with effective length
\leff:
\begin{align*}
  \vec\sang_{x,z} &= \gamma(t)a \int_{s_1}^{s_2} \bnvec[x,z] \dif s \approx \ga\cdot \bnvec[x,z](x(t),z(t)) \cdot\leff\\
  \text{and}\quad \vec\sang_{s} &= a \int_{s_1}^{s_2} \bnvec[s] \dif s \approx a\cdot \bnvec[s](x(t),z(t)) \cdot\leff
  \quad.
\end{align*}
The rotation angle is $\sang = |\vec\sang|$ and the rotation axis
is $\hat{r} = \vec\sang / \sang$.
%
The effective length \leff is taken from the \ele or \madx lattice.
The \textsc{Lorentz} factor $\gamma(t)$ and the trajectories $x(t)$ and $z(t)$ are
inserted for the time $t$ corresponding to the current tracking step.







\subsection{Magnetic Fields}
\label{sec:concept-b}
The magnetic field $\bnvec(x,z)$ of an element is calculated using the library \pal.
Currently, it supports elements of all types listed in \cref{sec:types}. The field vector
is
\begin{align*}
  \bnx(x,z) &= \knull{x} \;+\; k_1\cdot z \;+\; k_2\cdot x\cdot z \quad,\\
  \bnz(x,z) &= \knull{z} \;+\; k_1\cdot x \;+\; k_2\cdot \frac{x^2-z^2}{2} \quad,\\
  \bnlong(x,z) &= \knull{s} \quad.
\end{align*}
All parameters are imported from the \ele or \madx lattice automatically by \pal using the
\ele/\madx parameters listed in \cref{tab:pal-eleparas}.

\paragraph{Misalignments}
Currently, there are two kinds of magnet misalignments implemented in \pal and thus
supported by \polem:
\begin{itemize}
\item transversal displacements $\Delta x$ and $\Delta z$
\item rotations around the beam axis $s$
\end{itemize}
Both types are imported automatically from \ele and \madx respecting the particular sign
conventions of the particle tracking program. The corresponding \ele/\madx parameters are
also listed in \cref{tab:pal-eleparas}.




\subsection{Transversal Phase Space}
\label{sec:concept-traj}
Currently, there are three models of transversal trajectories $x(t)$ and $z(t)$ in \polem,
which can be selected with the configuration file entry \xmlinline{<trajectoryModel>} (see
\cref{sec:config-spintrk}):
\begin{itemize}
\item \xmlinline{closed orbit}: all particles move along the closed orbit imported from \ele or \madx
\item \xmlinline{simtool}: individual trajectories imported from \ele or \madx particle tracking for each particle
\item \xmlinline{oscillation}: simplified individual trajectories as oscillations
\end{itemize}

\paragraph{closed orbit}
Modeling transversal phase space simply by the closed orbit for all particles is
sufficient when simulating integer depolarizing resonances, because betatron tunes are
never integer and thus betatron motion does not contribute to integer resonances.

\paragraph{simtool}
Using realistic individual trajectories requires particle tracking over the full period
requested for spin tracking. I recommend to use the parallel version of \ele and to
compile \pal with \software{libSDDS1} support to allow \pal reading binary SDDS files without
converting them to ascii. Nevertheless a lot of RAM is recommended since the trajectories
are stored in memory completely by \software{libSDDS1}.

\paragraph{oscillation}
For simulations of intrinsic resonances without particle tracking, there is the simplified
\xmlinline{<trajectoryModel>} \xmlinline{oscillation}. It calculates the betatron
oscillation of particle $i$ as
\begin{equation}
  \label{eq:traj-oscillation}
  x_i(t) = \sqrt{\epsilon_{x,i} \beta_x} \cdot \cos\left(\omega_xt + \psi_{x,i}\right)
  \qquad\text{with}\quad
  \omega_x = \frac{2\pi Q_x}{T_\text{rev}}
\end{equation}
with the single particle emittance $\epsilon_{x,i}$, the beta function $\beta_x$, the
start phase $\psi_{x,i}$, the tune $Q_x$ and the revolution time $T_\text{rev}$. The same
equation is used for the vertical trajectory.
%
Beta functions, tunes and revolution time are imported automatically from the particle tracking
program. The start phases are chosen randomly (uniform distribution $[0,2\pi]$). The
beam emittances $\epsilon_x$ and $\epsilon_z$ have to be set in the configuration file.
Based on those, the single particle emittances are determined randomly (Gaussian
distribution around zero with $\sigma_x=\epsilon_x$).


\subsection{Longitudinal Phase Space}
\label{sec:concept-gamma}
Currently, there are two realistic models of longitudinal phase space in \polem, which can
be selected with the configuration file entry \xmlinline{<gammaModel>} (see
\cref{sec:config-spintrk}):
\begin{itemize}
\item \xmlinline{radiation}: implementation of stochastic synchrotron radiation within \polem
\item \xmlinline{simtool}: $\gamma_i(t)$ imported from \ele or \madx particle tracking
\end{itemize}
Furthermore, there are some simplified models of $\gamma(t)$ -- see
\cref{sec:config-spintrk}.

\paragraph{radiation}
This model is described in detail in \cite[section~4.2]{dr}. It
calculates the energy loss by radiating photons in dipole magnets based on the correct
statistical distributions of photon energy and photon number. The phase advance along
the accelerator is not tracked via the time of flight (dispersion), but approximated using
the momentum compaction factor (first and second order).

\paragraph{simtool}
The $\gamma_i(t)$ from \ele (or \madx) particle tracking may be used considering the same advises
given above for the transversal trajectories.



\subsubsection{Linear Energy Ramp}
A linear energy ramp can be set up in the \polem configuration file. When using
\xmlinline{<gammaModel>} \enquote{simtool} with \ele, the ramp is automatically
transmitted to \ele before starting the particle tracking. With \madx it has to be set up
manually in advance. For all other models the linear ramp is added directly to the
$\gamma_i(t)$.


\clearpage
\section{Configuration File Reference}
\label{sec:config}

\polem is configured with an \xml file. The individual entries are arranged in several
thematic groups like this:
\begin{xmlcode}
  <groupA>
    <entry1> value </entry1>
    <entry2> value </entry2>
  </groupA>
  <groupB>
    <entry42> value </entry42>
  </groupB>
\end{xmlcode}
The order of groups and entries is arbitrary. Many entries have default values and must
not be set. When \polem is executed, a configuration file with the actual values of all
entries is saved as \bashinline{currentconfig.pole}. All entries are
documented in the following.

\subsection{spintracking}
\label{sec:config-spintrk}

The group \xmlinline{<spintracking>} contains the setup of the spin tracking itself.\\[2mm]

\begin{configdoc}{t_start}{double}{\si{\s}}[0.0]
  The start time $t_\text{start}$ of the spin tracking
\end{configdoc}

\begin{configdoc}{t_stop}{double}{\si{\s}}
  The stop time $t_\text{stop}$ of the spin tracking
\end{configdoc}

\begin{configdoc}{E0}{double}{\si{\GeV}}
  Beam energy at the time $t=\SI{0}{\s}$
\end{configdoc}

\begin{configdoc}{dE}{double}{\si{\GeV\per\s}}
  Speed of the linear energy ramp
\end{configdoc}

\begin{configdoc}{Emax}{double}{\si{\GeV}}[\num{e10}]
  Maximum energy to confine the energy ramp.
  If the energy ramp reaches this energy, it is kept until the end of the tracking. Use a
  sufficiently large number do disable this feature (default).
\end{configdoc}

\begin{configdocgroup}{s_start}
  Initial value of the spin vector $\svec(t_\text{start})$. It is used for all spins.
  Thus, the tracking always starts with polarization $\pabs(t_\text{start}) = 1$. The
  three components of the vector are each set as a separate entry:
  
  \begin{configdoc}{x}{double}{}[0.0]
    Horizontal component \sx of the initial value
  \end{configdoc}

  \begin{configdoc}{z}{double}{}[1.0]
    Vertical component \sz of the initial value
  \end{configdoc}

  \begin{configdoc}{s}{double}{}[0.0]
    Longitudinal component \slong of the initial value
  \end{configdoc}
\end{configdocgroup}

\begin{configdoc}{numParticles}{unsigned int}{}[1]
  Number of particles (spin vectors) for the spin tracking
\end{configdoc}

\begin{configdoc}{dt_out}{double}{\si{\s}}[$(t_\text{stop}-t_\text{start})/1000$]
  Step width of the output of \pvec and \svec[i]
\end{configdoc}

\begin{configdoc}{outElement}{string}{}
  Name of a specific element in the lattice. If it is given, the output of \pvec and
  \svec[i] is printed only when passing this element. Thereby, the polarization can be
  observed at a specific position in a circular accelerator (e.g. a detector or extraction
  device). The output is written at the next passage after the time $t$ has increased by
  \xmlinline{<dt_out>}.
\end{configdoc}

\begin{configdoc}{gammaModel}{string}{}[radiation]
  Model for longitudinal beam dynamics $\gamma_i(t)$ during spin tracking. The two
  realistic and three simple models are discussed in \cite[chapter~4]{dr}.
  \begin{description}
  \item[\xmlinline{radiation}] This model is implemented in \polem and calculates
    stochastic emission of synchrotron radiation.
  \item[\xmlinline{simtool}] $\gamma_i(t)$ is imported from the particle tracking program,
    which is selected as \xmlinline{<simTool>} in the group \xmlinline{<palattice>} (see below).
    I recommend to use this model with \ele.
  \item[\xmlinline{linear}] Deactivation of longitudinal dynamics. All particles just
    follow the linear energy ramp: $\gamma_i(t) = \gcentral$.
  \item[\xmlinline{simtool+linear}] Sum of the two options above: The linear energy ramp
    is added to the imported $\gamma_i(t)$, which have been calculated for a constant
    energy.
  \item[\xmlinline{simtool_no_interpolation}] Similar to \xmlinline{simtool}, but faster
    and less precise. The recorded trajectory at the closest quadrupole is used directly
    instead of interpolating it to the current element position.
  \item[\xmlinline{offset}] Every particle gets an energy offset, which is constant over
    time: $\gamma_i(t) = \gcentral + \Delta\gamma_i$.
  \item[\xmlinline{oscillation}] Approximation of longitudinal dynamics by harmonic
    oscillations:\\$\gamma_i(t) = \gcentral + \Delta\gamma_i \cos(\ws t + \psi_i)$
  \end{description}
\end{configdoc}

\begin{configdoc}{trajectoryModel}{string}{}[closed orbit]
  Model for transverse particle trajectories $x_i(t)$ and
  $z_i(t)$ during spin tracking. To date, three models are available:
  \begin{description}
  \item[\xmlinline{closed orbit}] The closed orbit is used as trajectory for all
    particles. Thereby, all spins experience the same magnetic fields, which exclusively
    consist of revolution harmonic contributions (apart from explicitly time dependent
    fields). The closed orbit can be determined without particle tracking. This model is
    sufficient if the simulation of intrinsic resonances is not necessary.
  \item[\xmlinline{simtool}] The individual trajectories are imported from the particle
    tracking program selected as \xmlinline{<simTool>} in the group
    \xmlinline{<palattice>} below. The particle tracking is executed automatically.
  \item[\xmlinline{oscillation}] The individual trajectories are simplified as
    oscillations using tunes, beta functions and emittances as given in
    \cref{eq:traj-oscillation}. The emittances and tunes can be set in the group
    \xmlinline{<oscillation>} described below.
  \end{description}
\end{configdoc}

\begin{configdoc}{edgeFocussing}{bool}{}[false]
  Switch to enable horizontal magnetic fields at the edges of dipole magnets (edge
  focusing) during the spin tracking. Can be activated with \xmlinline{true} or
  \xmlinline{1} and deactivated with \xmlinline{false} or \xmlinline{0}.
\end{configdoc}




\subsection{palattice}
\label{sec:config-pal}
The group \xmlinline{<palattice>} contains the configuration of lattice import and
particle tracking.\\[2mm]

\begin{configdoc}{simTool}{string}{}
  Selection of the simulation tool for lattice import and particle tracking. Possible values:
  \xmlinline{elegant} or \xmlinline{madx}. I recommend using \ele since much more tracking
  parameters can be set automatically -- including number of particles, tracking duration
  and linear energy ramps. A \madx lattice can be converted automatically to an \ele
  lattice using \pal.
\end{configdoc}

\begin{configdoc}{mode}{string}{}
  Selection of import mode:
  \begin{description}
  \item[\xmlinline{online}] The simulation tool is configured and executed automatically.
    Typically, this mode should be used.
  \item[\xmlinline{offline}] In this mode the simulation tool is not executed. Instead,
    existing output files are imported. This can be helpful to avoid repeated execution of
    a particle tracking or to use \polem if \ele and \madx are not available.
  \end{description}
\end{configdoc}

\begin{configdoc}{file}{string}{}
  Lattice file for spin tracking. The file format has to be suitable for the program
  selected as \xmlinline{<simTool>}.
  Depending on \xmlinline{<mode>} varying files must be given:
  \begin{itemize}
  \item In \xmlinline{online} mode \xmlinline{<file>} is the lattice file
    (\bashinline{.lte} for \ele, usually \bashinline{.madx} for \madx).
  \item In \xmlinline{offline} mode \xmlinline{<file>} is the \bashinline{.param} output
    file (\ele) and the \bashinline{.twiss} output file (\madx) respectively.
  \end{itemize}
\end{configdoc}


\begin{configdoc}{saveGamma}{string}{}[]
  This option is meaningful only with \xmlinline{<gammaModel>} \xmlinline{simtool}. It
  allows to export the $\gamma_i(t)$, which are imported from the particle tracking, to
  text files. The option contains a list of particle numbers $i$ for the export.
  %
  The particle numbers are delimited by commas (e.g. \xmlinline{0,2,7}). Additionally,
  ranges of numbers can be given by hyphens (e.g. \xmlinline{0,3-6,8}). The text files are
  saved as \bashinline{gammaSimTool_i.dat} (particle number $i$).

  \textbf{Attention:} The \ele \bashinline{particleIDs} start at 1. This particle is
  assigned to particle number 0 in \polem.
\end{configdoc}

\begin{configdocgroup}{simToolRamp}
  If an \ele tracking is executed, the energy ramp configured in the group
  \xmlinline{<spintracking>} can be set for the particle tracking automatically. This is
  not implemented for \madx.

  \begin{configdoc}{set}{bool}{}[true]
    Switch for the energy ramp transfer to \ele. Can be activated with \xmlinline{true} or
    \xmlinline{1} and deactivated with \xmlinline{false} or \xmlinline{0}.
  \end{configdoc}

  \begin{configdoc}{steps}{unsigned int}{}[200]
    If the transfer is active, the ramp is calculated at discrete sampling points, which
    are then written to a \sdds file. Here, the number of sampling points can be changed.
    They are distributed equidistant from $t_\text{start}$ to $t_\text{stop}$.
  \end{configdoc}
\end{configdocgroup}

\clearpage
\begin{configdocgroup}{rfMagnets}
  These options can be used to configure the magnetic field $B$ of any lattice element as a
  time dependent alternating field (rf field)
  \begin{equation*}
    B(T) = B \cdot \cos\left( 2\pi \left[ Q_\text{rf,1}T + \frac{1}{2}\Delta
        Q_\text{rf} T^2 \right] \right)
  \end{equation*}
  as a function of the turn $T$.
  The setup only affects the spin tracking and not the particle tracking with \ele or
  \madx. In addition, the field has to be set up in \ele/\madx, if the influence of an rf
  field on beam dynamics should be included.

  An arbitrary number of elements can be set up with an rf field since all
  following options allow for giving multiple values as a comma separated list. All
  options must have the same number of entries, which are then assigned to the
  corresponding elements in \xmlinline{<elements>}.\\[1mm]

  \begin{configdoc}{elements}{string}{}[]
    Name of the element for rf field. [multiple elements as comma separated list]
  \end{configdoc}

  \begin{configdoc}{Q1}{string}{}[]
    Frequency of the rf field at the beginning of the spin tracking (turn $T=1$) normalized to
    the revolution frequency (tune $Q_\text{rf,1}$). [for multiple elements as comma separated list]
  \end{configdoc}

  \begin{configdoc}{dQ}{string}{}[]
    Frequency change per turn (frequency sweep) -- also as a tune $\Delta Q_\text{rf}$.
    [for multiple elements as comma separated list]
  \end{configdoc}

  \begin{configdoc}{period}{string}{}[]
    Duration of the frequency sweep in turns. After this period the sweep starts again.
    [for multiple elements as comma separated list]
  \end{configdoc}
\end{configdocgroup}



\subsection{radiation}
\label{sec:config-rad}
This group is for configuration of the \xmlinline{<gammaModel>} \xmlinline{radiation},
which is a model of longitudinal dynamics implemented in \polem to allow for a stochastic
synchrotron radiation model without import from a particle tracking program. If another
\xmlinline{<gammaModel>} is selected, the whole group can be ignored.


\begin{configdoc}{seed}{int}{}[\textit{random}]
  The seed of the random number generator, which is used for the initial particle
  distribution in phase space and the stochastic photon emission.
\end{configdoc}

\begin{configdocgroup}{savePhaseSpace}
  Here, particle numbers $i$ can be chosen for a plain text export of longitudinal phase
  space $(\phf,\gamma)$ -- analog to \xmlinline{<saveGamma>} in group
  \xmlinline{<palattice>}.

  \begin{configdoc}{list}{string}{}[]
    A list of particle numbers $i$ for the export. The particle numbers are delimited by
    commas (e.g. \xmlinline{0,2,7}). Additionally, ranges of numbers can be given by
    hyphens (e.g. \xmlinline{0,3-6,8}). The text files are saved as
    \bashinline{longPhaseSpace_i.dat} (particle number $i$).
  \end{configdoc}

  \begin{configdoc}{elementName}{string}{}[]
    The name of an element in the lattice. The phase space coordinates are recorded at the
    position of this element. This option must be set to use \xmlinline{<savePhaseSpace>}.
    Else no output is written.
  \end{configdoc}
\end{configdocgroup}

\clearpage
\begin{configdocgroup}{startDistribution}
  \polem calculates the initial phase space coordinates of all particles so that the
  $\phf_i$ and the $\gamma_i$ are Gaussian distributed and average and width of the
  distribution correspond to the radiation equilibrium. The following two options can be
  used to vary the width  of these distributions.

  \begin{configdoc}{sigmaPhaseFactor}{double}{}[1.0]
    Factor to scale the width of the Gaussian distribution of the phases $\phf_i$. A value
    of 1 corresponds to the radiation equilibrium.
  \end{configdoc}

  \begin{configdoc}{sigmaGammaFactor}{double}{}[1.0]
    Factor to scale the width of the Gaussian distribution of the particle energies
    $\gamma_i$. A value of 1 corresponds to the radiation equilibrium.
  \end{configdoc}
\end{configdocgroup}


The following options are physics parameters of the accelerator, which are required by the
synchrotron radiation model. They are determined by the particle tracking program and
imported automatically if the corresponding option is not set here (default value
\num{0.0}). Thus, these options can be used to enforce other values.\\[2mm]

\begin{configdoc}{momentum_compaction_factor}{double}{}[0.0]
  the momentum compaction factor \alphac
\end{configdoc}

\begin{configdoc}{momentum_compaction_factor_2}{double}{}[0.0]
   the second order momentum compaction factor \alphactwo
\end{configdoc}

\begin{configdoc}{overvoltage_factor}{double}{}[0.0]
  the overvoltage factor $q$
\end{configdoc}

\begin{configdoc}{harmonic_number}{unsigned int}{}[0]
  the harmonic number $h$
\end{configdoc}

\begin{configdoc}{bending_radius}{double}{\si{\m}}[0.0]
  the average bending radius $R$ of the dipole magnets
\end{configdoc}

\begin{configdoc}{longitudinal_damping_partition_number}{double}{}[0.0]
  the longitudinal damping partition number $J_s$
\end{configdoc}


\subsection{oscillation}
\label{sec:config-oscil}
This group is for configuration of the \xmlinline{<trajectoryModel>}
\xmlinline{oscillation}, which is described in \cref{sec:concept-traj}. If another
\xmlinline{<trajectoryModel>} is selected, the whole group can be ignored.


\begin{configdocgroup}{emittance}
  The amplitude of the betatron oscillations depends on the beam emittance, which has to
  be set here. It is not imported from \ele or \madx.\\[2mm]
  
  \begin{configdoc}{x}{double}{\si{\m\radian}}
  horizontal emittance $\epsilon_x$
  \end{configdoc}
  \begin{configdoc}{z}{double}{\si{\m\radian}}
    vertical emittance $\epsilon_z$
  \end{configdoc}
\end{configdocgroup}

\clearpage
\begin{configdocgroup}{tune}
  The frequency of the betatron oscillations is given by the tunes.
  They are determined by the particle tracking program and
  imported automatically if the corresponding option is not set here (default value
  \num{0.0}). Thus, these options can be used to enforce other values.\\[2mm]
  
  \begin{configdoc}{x}{double}{\si{\m\radian}}[0.0]
  horizontal tune $Q_x$
  \end{configdoc}
  \begin{configdoc}{z}{double}{\si{\m\radian}}[0.0]
    vertical tune $Q_z$
  \end{configdoc}
\end{configdocgroup}



\subsection{resonancestrengths}
\label{sec:config-resstr}
This group is for configuration of the estimation of resonance strengths of depolarizing
resonances. This special mode can be activated by the command line option \bashinline{-R}.
If \polem is used for spin tracking, the whole group can be omitted.\\[2mm]

\begin{configdocgroup}{spintune}
  The resonance strength is calculated for discrete values of the spin tune \ga and
  written to the text file \bashinline{resonance-strengths.dat}. Here, the output range and
  step width can be set.

  \begin{configdoc}{min}{double}{}[0.0]
    minimum spin tune \ga
  \end{configdoc}

  \begin{configdoc}{max}{double}{}[10.0]
    maximum spin tune \ga
  \end{configdoc}

  \begin{configdoc}{step}{double}{}[1.0]
    step width $\Delta\ga$ for calculation and output of the spin tune
  \end{configdoc}
\end{configdocgroup}

\begin{configdoc}{turns}{unsigned int}{}[0]
  If the resonance strengths are calculated for non integer \ga, multiple particles with
  their individual trajectories have to be taken into account and the resulting resonance
  strength is the average over all particles. For this purpose, the trajectories are
  imported from the particle tracking program as it is configured in the group
  \xmlinline{<palattice>}. The number of particles is taken from the option
  \xmlinline{<numParticles>} (Gruppe \xmlinline{<spintracking>}).

  Here the number of turns $N_u$ for the trajectories can bes set. If the default value 0
  is used, the number of turns is set automatically according to the minimum number
  required for the chosen frequency resolution \xmlinline{<step>}: $N_u = 1/\Delta\ga$.
\end{configdoc}




\appendix
\clearpage
\section{\ele Installation Guide}
\label{sec:ele-install}

\ele packages for many operating systems can be downloaded at \cite{elegant-download}.
%
Additionally you have to download the \bashinline{defns.rpn} file and store its path in
the environment variable \bashinline{RPN_DEFNS}. E.g. add the following line to your
\bashinline{~/.bashrc} file:
\begin{bashcode}
  export RPN_DEFNS = '/path/to/defns.rpn'
\end{bashcode}

If there is no \ele package for your system, you can download the \software{Build-AOP-RPMs}
script instead which automatically builds all desired programs from source.
%
If you want to use \software{Pelegant}, first install \software{mpicc}. Under Ubuntu it is
in the package \bashinline{mpi-default-dev}.
%
Now go to the folder with the \software{Build-AOP-RPMs} script and run
\begin{bashcode}
  sudo ./Build-AOP_RPMs
\end{bashcode}
Accept the programs you want to install by hitting \bashinline{y} and decline all others
with \bashinline{n}. For \pal and \polem you need
\begin{itemize}
\item SDDS (Not SDDSEpics)
\item elegant
\item (Pelegant recommended)
\end{itemize}
For \software{Pelegant} the script asks for the path of \software{mpicc}. You can
determine it by calling \bashinline{which mpicc}. The script expects the directory (without
\bashinline{mpicc} at the end).
%
The scripts downloads the source code of all programs, builds them and creates rpm
packages. This will take some time. Finally, these packages can be found in
\bashinline{~/rpmbuild/RPMs/} (additional subfolder depending on the architecture).
Install each package. Under Ubuntu rpm package files can be installed via
\begin{bashcode}
  sudo alien -i packagename.rpm
\end{bashcode}

\software{Pelegant} is called via \bashinline{mpirun -n 4 Pelegant}, where the argument of
the option \bashinline{-n} is the number of threads used for tracking.
%
To use \software{Pelegant} for the automatic \ele execution by \polem, you have to modify
the \ele command in the \bashinline{config.hpp} file of \pal and build and install \pal
again (see \pal README for details).


\clearpage
\section{Supported \ele/\madx Element Types and Parameters}
\label{sec:types}

\begin{table}[h]
  \centering
  \begin{tabular}{lll}
    \toprule
    \pal & \ele & \madx \\
    \midrule
    Dipole & CSBEND & SBEND \\
    Corrector & (H,V)KICK oder KICKER & (H,V)KICKER, \\
    Solenoid & SOLE & SOLENOID \\
    Quadrupole & KQUAD & QUADRUPOLE \\
    Sextupole & KSEXT & SEXTUPOLE \\
    Multipole & MULT & MULTIPOLE \\
    Marker & MARK & MARKER \\
    Monitor & MONI & MONITOR \\
    Cavity & RFCA & RFCAVITY \\
    Rcollimator & RCOL & RCOLLIMATOR \\
    Drift & DRIF & DRIFT \\
    \bottomrule
  \end{tabular}
  \caption{Assignment of lattice element types in the \pal library to the types in \ele
    and \madx. Other types are ignored during lattice import in \polem.}
  \label{tab:pal-eletypes}
\end{table}

\begin{table}[h]
  \centering
  \begin{tabular}{lll}
    \toprule
    \pal & \ele & \madx \\
    \midrule
    name & ElementName & NAME \\
    length & L & L \\
    k0.x & sin([V]KICK)/L &  sin(VKICK)/L \\
    k0.z & ANGLE/L - sin([H]KICK)/L & ANGLE/L - sin(HKICK)/L \\
    k0.s & KS &  KSI/L\\
    k1 & K1 & K1L/L \\
    k2 & K2 &  K2L/L\\
    e1 & E1 & E1 \\
    e2 & E2 & E2 \\
    tilt & TILT & TILT + DPSI \\
    displacement.x & DX & DX \\
    displacement.z & DY & DY \\
    halfWidth.x & X\_MAX & APER\_1 \\
    halfWidth.z & Y\_MAX & APER\_2 \\
    volt & VOLT & VOLT \\
    freq &  FREQ & FREQ \\
    \bottomrule
  \end{tabular}
  \caption{Assignment of element parameters in \pal
    %(member variables of \cppinline{pal::AccElement})
    to parameters in \ele and \madx. Special cases: For \ele k0.x and k0.z have to be
    distinguished between element types KICK (parameters HKICK und VKICK) and HKICK und
    VKICK (parameter KICK). Aperture is only imported from \madx if APERTYPE has the value
    RECTANGLE. }
  \label{tab:pal-eleparas}
\end{table}

\clearpage
\printbibliography[heading=bibintoc]

\end{document}



%%% Local Variables: 
%%% mode: Latex
%%% LaTeX-command: "latex -shell-escape"
%%% End: 
