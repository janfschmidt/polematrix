%% polematrix documentation
%% Copyright (C) 2017 Jan Felix Schmidt <janschmidt@mailbox.org>
%%   
%% This program is free software: you can redistribute it and/or modify
%% it under the terms of the GNU General Public License as published by
%% the Free Software Foundation, either version 3 of the License, or
%% (at your option) any later version.
%%
%% This program is distributed in the hope that it will be useful,
%% but WITHOUT ANY WARRANTY; without even the implied warranty of
%% MERCHANTABILITY or FITNESS FOR A PARTICULAR PURPOSE.  See the
%% GNU General Public License for more details.
%%
%% You should have received a copy of the GNU General Public License
%% along with this program.  If not, see <http://www.gnu.org/licenses/>.

\documentclass[a4paper]{scrartcl}
\usepackage[english]{babel}
\usepackage[utf8]{inputenc}
\usepackage[T1]{fontenc}
\usepackage[margin=2.5cm]{geometry}
\usepackage{siunitx}
\usepackage{csquotes}
\usepackage{booktabs}
\usepackage[colorlinks]{hyperref}
\usepackage[backend=biber,style=alphabetic]{biblatex}
\usepackage{todonotes}

\usepackage{polem_defs}
\usepackage{cleveref}

\author{Jan Felix Schmidt \textless janschmidt@mailbox.org\textgreater}
\title{polematrix}
\subtitle{a spin tracking code for electron accelerators using elegant}
\date{documentation for version 0.99}

\hypersetup{
  pdftitle={polematrix documentation},
  pdfauthor={Jan Felix Schmidt}
}

\bibliography{polem}

\setcounter{tocdepth}{2}

\begin{document}
\maketitle

\begin{abstract}
  \polem is a spin tracking code. It calculates the motion of the spin of ultra
  relativistic electrons in particle accelerators. It implements a simple matrix tracking
  as multi-thread \cpp application. The input can be any lattice file from \ele or
  \madx. All required particle trajectories are also imported from the established
  particle tracking codes \ele or \madx.

  \polem includes synchrotron radiation via import from \ele or an own implementation of
  longitudinal phasespace. A short description of \polem is given in the IPAC'16
  contribution \cite{IPAC16-decoh} and in my phd thesis, which will be available soon (in
  German).
  
  Do not hesitate to contact me if you have any questions and please report bugs.
\end{abstract}

\tableofcontents
\clearpage



\section{Usage}
\label{sec:usage}

\polem is called via
\begin{bashcode}
  polematrix [options] [CONFIGURATION FILE]
\end{bashcode}
where \bashinline{[options]} are the command line options shown in
\cref{tab:polem-options}. All parameters of the spin tracking can be set up in an \xml
configuration file (\bashinline{[CONFIGURATION FILE]}) which is described in
\cref{sec:config}.


\begin{table}[h]
  \centering
  \begin{tabular}{ll}
    \toprule
    \bashinline{-h [ --help ]}              &  display this help message \\
    \bashinline{-V [ --version ]}           &  display version \\
    \bashinline{-T [ --template ]}          &  create config file template (\bashinline{template.pole}) and quit \\
    \bashinline{-R [ --resonance-strengths ]} &  estimate resonance strengths instead of spin tracking \\
    \midrule
    \bashinline{-t [ --threads ] arg (=all)}     &  set number of threads used for tracking \\
    \bashinline{-o [ --output-path ] arg (=.)}   &  path for output files \\
    \bashinline{-v [ --verbose ]}            &  activate additional status output \\
    \bashinline{-n [ --no-progressbar ]}     &  do not show progress bar during tracking\\
                                            &  (e.g. if output is redirected to a log file)\\
    \bashinline{-a [ --all ]}                &  write additional output files (e.g. lattice und orbit) \\
    \bashinline{-s [ --spintune ] arg}       &  in resonance strengths mode (\bashinline{-R}):\\
                                            &  calculate for given spin tune only\\
    \bottomrule
  \end{tabular}
  \caption{Command line options of \polem.}
  \label{tab:polem-options}
\end{table}




\section{Installation}
\label{sec:installation}

\polem has been tested only on Ubuntu/Debian Linux with the GCC compiler so far.

\subsection{Dependencies}
\label{sec:dependencies}

\polem requires the following programs and libraries:
\begin{itemize}
\item GCC compiler
\item CMake
\item Gnu Scientific Library (GSL) \cite{gsl}
\item Boost program options
\item Boost filesystem
\item Boost property tree \cite{boost-pt}
\item Boost random \cite{boost-random}
\item Armadillo library \cite{arma}
\item \pal library \cite{palattice}
\end{itemize}
\pal is available on github. The source code can be downloaded at \cite{palattice} and
compiled and installed as described in the enclosed README file.
%
All other dependencies may be in the package repositories of your Linux distribution. On
Ubuntu you can install all required packages with the following command:
\begin{bashcode}
  sudo apt-get install g++ cmake libgsl-dev libboost-dev libboost-program-options-dev libboost-filesystem-dev libarmadillo-dev
\end{bashcode}
Additionally, \polem relies on the established particle
tracking program \ele (or \madx), whose lattice files and tracking results are used for
spin tracking.  \ele (and \madx) can be configured and execute automatically by \polem using
\pal. The installation of \ele and \madx is described in \cref{sec:elemadx}.


    
\subsection{Build and Install \polem}
\label{sec:build}
\polem is compiled and installed using \software{CMake} with the following commands:
\begin{bashcode}
  cd polematrix/build
  cmake ../
  make
  sudo make install
\end{bashcode}
The default install path is \bashinline{/usr/local/bin}. It can be changed by setting the
\bashinline{CMAKE_INSTALL_PREFIX} variable in \bashinline{polematrix/CMakeLists.txt}.
%
To uninstall \polem run
\begin{bashcode}
  cd polematrix/build
  sudo make uninstall
\end{bashcode}



\subsection{Underlying Particle Tracking Programs}
\label{sec:elemadx}

I recommend to use \polem with \ele since much more tracking parameters can be set
automatically -- including number of particles (starting with a Gaussian profile in
radiation equilibrium), tracking duration and linear energy ramps. If you have only a
lattice file in \madx format, you can convert it automatically to an \ele lattice
using the program \software{convertlattice} which is included in \pal \cite{palattice}.

\subsubsection{\ele}
\label{sec:ele}
The particle tracking program \ele \cite{elegant} is developed at the Advanced Photon
Source at Argonne National Laboratory. Packages for many operating systems can be
downloaded at \cite{elegant-download}. Also install the \software{SDDSToolKit}, which is
needed to access the binary output files of \ele.
%
If there is no package for your system, you can download the \bashinline{Build-AOP-RPMs}
script instead which automatically builds all desired programs from source (see the guide
in \cref{sec:ele-install}).
%
There is a parallel version of \ele, called \software{Pelegant}, which speeds up the
tracking \cite{pelegant} (installation hints in \cref{sec:ele-install}). The \ele manual
can be found at \cite{elegant-manual}.

\subsubsection{\madx}
\label{sec:madx}
The particle tracking program \madx is developed at CERN and can be downloaded from the
website \cite{madx}, which also provides the \madx manual.





\section{Introduction to the Concept of \polem}
\label{sec:concept}




\section{Configuration File Reference}
\label{sec:config}

\polem is configured with an \xml file. The individual entries are arranged in several
thematic groups like this:
\begin{xmlcode}
  <groupA>
    <entry1> value </entry1>
    <entry2> value </entry2>
  </groupA>
  <groupB>
    <entry42> value </entry42>
  </groupB>
\end{xmlcode}
The order of groups and entries is arbitrary. Many entries have default values and must
not be set. When \polem is executed, a configuration file with the actual values of all
entries is saved as \bashinline{currentconfig.pole}. All entries are
documented in the following.

\subsection{spintracking}

The group \xmlinline{<spintracking>} contains the setup of the spin tracking itself.\\[2mm]

\begin{configdoc}{t_start}{double}{\si{\s}}[0.0]
  The start time $t_\text{start}$ of the spin tracking
\end{configdoc}

\begin{configdoc}{t_stop}{double}{\si{\s}}
  The stop time $t_\text{stop}$ of the spin tracking
\end{configdoc}

\begin{configdoc}{E0}{double}{\si{\GeV}}
  Beam energy at the time $t=\SI{0}{\s}$
\end{configdoc}

\begin{configdoc}{dE}{double}{\si{\GeV\per\s}}
  Speed of the linear energy ramp
\end{configdoc}

\begin{configdoc}{Emax}{double}{\si{\GeV}}[\num{e10}]
  Maximum energy to confine the energy ramp.
  If the energy ramp reaches this energy, it is kept until the end of the tracking. Use a
  sufficiently large number do disable this feature (default).
\end{configdoc}

\begin{configdocgroup}{s_start}
  Initial value of the spin vector $\svec(t_\text{start})$. It is used for all spins.
  Thus, the tracking always starts with polarization $\pabs(t_\text{start}) = 1$. The
  three components of the vector are each set as a separate entry:
  
  \begin{configdoc}{x}{double}{}[0.0]
    Horizontal component \sx of the initial value
  \end{configdoc}

  \begin{configdoc}{z}{double}{}[1.0]
    Vertical component \sz of the initial value
  \end{configdoc}

  \begin{configdoc}{s}{double}{}[0.0]
    Longitudinal component \slong of the initial value
  \end{configdoc}
\end{configdocgroup}

\begin{configdoc}{numParticles}{unsigned int}{}[1]
  Number of particles (spin vectors) for the spin tracking
\end{configdoc}

\begin{configdoc}{dt_out}{double}{\si{\s}}[$(t_\text{stop}-t_\text{start})/1000$]
  Step width of the output of \pvec and \svec[i]
\end{configdoc}

\begin{configdoc}{gammaModel}{string}{}[radiation]
  Model for longitudinal beam dynamics $\gamma_i(t)$ during spin tracking. The two
  realistic and three simple models are discussed in \cite[chapter~4]{dr}.
  \todo[inline]{add doc section about gammaModels}
  \begin{description}
  \item[\xmlinline{radiation}] This model is implemented in \polem and calculates
    stochastic emission of synchrotron radiation.
  \item[\xmlinline{simtool}] $\gamma_i(t)$ is imported from the particle tracking program,
    which is selected as \xmlinline{<simTool>} in the group \xmlinline{<palattice>} (see below).
    I recommend to use this model with \ele.
  \item[\xmlinline{linear}] Deactivation of longitudinal dynamics. All particles just
    follow the linear energy ramp: $\gamma_i(t) = \gcentral$.
  \item[\xmlinline{offset}] Every particle gets an energy offset, which is constant over
    time: $\gamma_i(t) = \gcentral + \Delta\gamma_i$.
  \item[\xmlinline{oscillation}] Approximation of longitudinal dynamics by harmonic
    oscillations:\\$\gamma_i(t) = \gcentral + \Delta\gamma_i \cos(\ws t + \psi_i)$
  \end{description}
\end{configdoc}

\begin{configdoc}{trajectoryModel}{string}{}[closed orbit]
  Model for transverse particle trajectories $x_i(t)$ and
  $z_i(t)$ during spin tracking. To date, two models are available:
  \begin{description}
  \item[\xmlinline{closed orbit}] The closed orbit is used as trajectory for all
    particles. Thereby, all spins experience the same magnetic fields, which exclusively
    consist of revolution harmonic contributions (apart from explicitly time dependent
    fields). The closed orbit can be determined without particle tracking. This model is
    sufficient if the simulation of intrinsic resonances is not necessary.
  \item[\xmlinline{trajectory}] The individual trajectories are imported from the particle
    tracking program selected as \xmlinline{<simTool>} in the group
    \xmlinline{<palattice>} below. The particle tracking is executed automatically.
  \end{description}
\end{configdoc}

\begin{configdoc}{edgeFocussing}{bool}{}[false]
  Switch to enable horizontal magnetic fields at the edges of dipole magnets (edge
  focusing) during the spin tracking. Can be activated with \xmlinline{true} or
  \xmlinline{1} and deactivated with \xmlinline{false} or \xmlinline{0}. \todo[inline]{add
    sec about angle def}
\end{configdoc}




\subsection{palattice}
The group \xmlinline{<palattice>} contains the configuration of lattice import and
particle tracking.\\[2mm]

\begin{configdoc}{simTool}{string}{}
  Selection of the simulation tool for lattice import and particle tracking. Possible values:
  \xmlinline{elegant} or \xmlinline{madx}. I recommend using \ele since much more tracking
  parameters can be set automatically -- including number of particles, tracking duration
  and linear energy ramps. A \madx lattice can be converted automatically to an \ele
  lattice using \pal.
\end{configdoc}

\begin{configdoc}{mode}{string}{}
  Selection of import mode:
  \begin{description}
  \item[\xmlinline{online}] The simulation tool is configured and executed automatically.
    Typically, this mode should be used.
  \item[\xmlinline{offline}] In this mode the simulation tool is not executed. Instead,
    existing output files are imported. This can be helpful to avoid repeated execution of
    a particle tracking or to use \polem if \ele and \madx are not available.
  \end{description}
\end{configdoc}

\begin{configdoc}{file}{string}{}
  Lattice file for spin tracking. The file format has to be suitable for the program
  selected as \xmlinline{<simTool>}.
  Depending on \xmlinline{<mode>} varying files must be given:
  \begin{itemize}
  \item In \xmlinline{online} mode \xmlinline{<file>} is the lattice file
    (\bashinline{.lte} for \ele, usually \bashinline{.madx} for \madx).
  \item In \xmlinline{offline} mode \xmlinline{<file>} is the \bashinline{.param} output
    file (\ele) and the \bashinline{.twiss} output file (\madx) respectively.
  \end{itemize}
\end{configdoc}


\begin{configdoc}{saveGamma}{string}{}[]
  This option is meaningful only with \xmlinline{<gammaModel>} \xmlinline{simtool}. It
  allows to export the $\gamma_i(t)$, which are imported from the particle tracking, to
  text files. The option contains a list of particle numbers $i$ for the export.
  %
  The particle numbers are delimited by commas (e.g. \xmlinline{0,2,7}). Additionally,
  ranges of numbers can be given by hyphens (e.g. \xmlinline{0,3-6,8}). The text files are
  saved as \bashinline{gammaSimTool_i.dat} (particle number $i$).

  \textbf{Attention:} The \ele \bashinline{particleIDs} start at 1. This particle is
  assigned to particle number 0 in \polem.
\end{configdoc}

\begin{configdocgroup}{simToolRamp}
  If an \ele tracking is executed, the energy ramp configured in the group
  \xmlinline{<spintracking>} can be set for the particle tracking automatically. This is
  not implemented for \madx.

  \begin{configdoc}{set}{bool}{}[true]
    Switch for the energy ramp transfer to \ele. Can be activated with \xmlinline{true} or
    \xmlinline{1} and deactivated with \xmlinline{false} or \xmlinline{0}.
  \end{configdoc}

  \begin{configdoc}{steps}{unsigned int}{}[200]
    If the transfer is active, the ramp is calculated at discrete sampling points, which
    are then written to a \sdds file. Here, the number of sampling points can be changed.
    They are distributed equidistant from $t_\text{start}$ to $t_\text{stop}$.
  \end{configdoc}
\end{configdocgroup}

\begin{configdocgroup}{rfMagnets}
  These options can be used to configure the magnetic fields of any lattice element as a
  time dependent alternating field.\todo{formula} The setup only affects the spin tracking
  and not the particle tracking with \ele or \madx. In addition, the field has to be set
  up in \ele/\madx, if the influence of an alternating field on beam dynamics should be
  included.

  An arbitrary number of elements can be set up with an alternating field since all
  following options allow for giving multiple values as a comma separated list. All
  options must have the same number of entries, which are then assigned to the
  corresponding elements in \xmlinline{<elements>}.\\[1mm]

  % \begin{configdoc}{elements}{string}{}[]
  %   Namen des Elements für das das Wechselfeld aktiviert werden soll (mehrere
  %   Elemente mit Kommata getrennt).
  % \end{configdoc}

  % \begin{configdoc}{Q1}{string}{}[]
  %   Frequenz des Wechselfeldes zu Beginn des Trackings (Umlauf 1) normiert auf die
  %   Umlauffrequenz, also als einheitenloser Arbeitspunkt (oder für mehrere Elemente mit
  %   Kommata getrennt).
  % \end{configdoc}

  % \begin{configdoc}{dQ}{string}{}[]
  %   Änderung der Frequenz pro Umlauf (Frequenz-\foreignquote{american}{sweep}) --
  %   ebenfalls als einheitenloser Arbeitspunkt (für mehrere Elemente mit Kommata getrennt).
  % \end{configdoc}

  % \begin{configdoc}{period}{string}{}[]
  %   Länge des Frequenz-\foreignquote{american}{sweeps} in Umläufen. Danach beginnt der
  %   \foreignquote{american}{sweep} von vorne (für mehrere Elemente mit Kommata getrennt).
  % \end{configdoc}
\end{configdocgroup}



% \subsection{radiation}
% In dieser Gruppe wird das in \polem implementierte Modell der longitudinalen Strahldynamik
% unter Einbeziehung von \synrad konfiguriert. Dementsprechend ist die gesamte Gruppe nur
% dann relevant, wenn unter \xmlinline{<spintracking>} der \xmlinline{<gammaModel>}
% \xmlinline{radiation} gewählt ist. Andernfalls kann die Gruppe vollständig weggelassen werden.\\[2mm]

% \begin{configdoc}{seed}{int}{}[zufällig]
%   Der \foreignquote{american}{seed} für den Zufallszahlengenerator, der die
%   Startverteilung der Teilchen und die stochastische Abstrahlung bestimmt. Durch Angabe
%   des selben Wertes bei mehreren Simulationen können die Ergebnisse exakt reproduziert werden. 
% \end{configdoc}

% \begin{configdocgroup}{savePhaseSpace}
%   Analog zu \xmlinline{<saveGamma>} in der Gruppe \xmlinline{<palattice>} können hier
%   Teilchennummern $i$ gewählt werden, für die die Koordinate im longitudinalen
%   Phasenraum $(\phf,\gamma)$ während des Spintrackings in einer Textdatei aufgezeichnet
%   werden soll.

%   \begin{configdoc}{list}{string}{}[]
%     Diese Option enthält eine Liste der Teilchennummern $i$, für die die Aufgezeichnung
%     aktiviert werden soll. Die Teilchennummern werden durch Kommata getrennt (z.B.
%     \xmlinline{0,2,7}). Außerdem können Bereiche durch Bindestriche angegeben werden (z.B.
%     \xmlinline{0,3-6,8}). Die Dateien werden unter dem Namen
%     \bashinline{longPhaseSpace_i.dat} (mit Teilchennummer $i$) gespeichert. Die Ausgabe
%     erfolgt in jedem Umlauf an der Position des unter \xmlinline{<elementName>}
%     angegebenen Elements.
%   \end{configdoc}

%   \begin{configdoc}{elementName}{string}{}[]
%     Der Name des Elements im Lattice, an dessen Position die Phasenraumkoordinaten
%     aufgezeichnet werden sollen. Die Option muss gesetzt werden, wenn
%     \xmlinline{<savePhaseSpace>} verwendet werden soll. Sonst erfolgt keine Ausgabe.
%   \end{configdoc}
% \end{configdocgroup}

% \begin{configdocgroup}{startDistribution}
%   \polem berechnet die Startkoordinaten der Teilchen im longitudinalen Phasenraum zum
%   Beginn des Spintrackings ($t_\text{start}$) automatisch so, dass die $\phf_i$ und
%   $\gamma_i$ \textsc{gauß}verteilt sind und Schwerpunkt und Breite der Verteilungen dem
%   Strahlungsgleichgewicht entsprechen (siehe \cref{sec:radiation-model}). Mit diesen
%   beiden Optionen können die Startwerte der Verteilungen in ihrer Breite verändert werden.
%   Dann ist beispielsweise die Strahlungsdämpfung erkennbar (siehe
%   \cref{fig:radmodel-damping}).

%   \begin{configdoc}{sigmaPhaseFactor}{double}{}[1.0]
%     Faktor zur Skalierung der Breite der \textsc{Gauß}verteilung der Phasen $\phf_i$. Der
%     Wert 1 entspricht dem Strahlungsgleichgewicht.
%   \end{configdoc}

%   \begin{configdoc}{sigmaGammaFactor}{double}{}[1.0]
%     Faktor zur Skalierung der Breite der \textsc{Gauß}verteilung der Teilchenenergien
%     $\gamma_i$. Der Wert 1 entspricht dem Strahlungsgleichgewicht.
%   \end{configdoc}
% \end{configdocgroup}

% Bei den folgenden Optionen handelt es sich um physikalische Parameter des Beschleunigers,
% die zur Berechnung der \sylia benötigt werden. Sie werden von dem verwendeten
% Teilchentracking-Programm berechnet und automatisch übernommen, wenn die jeweilige Option
% nicht angegeben ist (Standardwert \num{0.0}). Die Optionen können verwendet werden um
% abweichende Werte zu erzwingen.\\[2mm]

% \begin{configdoc}{momentum_compaction_factor}{double}{}[0.0]
%   der \momcomp \alphac
% \end{configdoc}

% \begin{configdoc}{momentum_compaction_factor_2}{double}{}[0.0]
%     der \momcomp zweiter Ordnung \alphactwo
% \end{configdoc}

% \begin{configdoc}{overvoltage_factor}{double}{}[0.0]
%   der Überspannungsfaktor $q$
% \end{configdoc}

% \begin{configdoc}{harmonic_number}{unsigned int}{}[0]
%   die Harmonischenzahl $h$
% \end{configdoc}

% \begin{configdoc}{bending_radius}{double}{\si{\m}}[0.0]
%   der mittlere Ablenkradius $R$ der Dipolmagnete
% \end{configdoc}

% \begin{configdoc}{longitudinal_damping_partition_number}{double}{}[0.0]
%   die longitudinale Dämpfungszahl $J_s$
% \end{configdoc}



% \subsection{resonancestrengths}
% Diese Gruppe dient der Einstellung der Berechnung von Resonanzstärken depolarisierender
% Resonanzen, die über die Kommandozeilenoption \bashinline{-R} aktiviert wird. Soll \polem
% für das Spintracking verwendet werden, kann die gesamte Gruppe weggelassen werden. Die
% Implementation der Berechnung von Resonanzstärken ist in \cref{sec:resstrengths}
% beschrieben.\\[2mm]

% \begin{configdocgroup}{spintune}
%   Die Resonanzstärke wird für diskrete Werte des \stune[es] \ga berechnet und in die
%   Textdatei \bashinline{resonance-strengths.dat} geschrieben. Hier können Bereich und
%   Schrittweie der Berechnung und Ausgabe konfiguriert werden.

%   \begin{configdoc}{min}{double}{}[0.0]
%     minimaler \stune \ga
%   \end{configdoc}

%   \begin{configdoc}{max}{double}{}[10.0]
%     maximaler \stune \ga
%   \end{configdoc}

%   \begin{configdoc}{step}{double}{}[1.0]
%     Schrittweite $\Delta\ga$ für Berechnung und Ausgabe des \stune[es]
%   \end{configdoc}
% \end{configdocgroup}

% \begin{configdoc}{turns}{unsigned int}{}[0]
%   Wenn Resonanzstärken nicht nur für ganzzahlige \ga berechnet werden sollen, müssen die
%   individuellen Trajektorien der Teilchen berücksichtigt werden und die Resonanzstärken über alle
%   Teilchen gemittelt werden (siehe \cref{sec:resstrengths}). Dazu wird der Import der
%   Trajektorien aus dem Teilchentracking analog zum Spintracking in der Gruppe
%   \xmlinline{<palattice>} konfiguriert. Die Anzahl der Teilchen, wird aus der Option
%   \xmlinline{<numParticles>} (Gruppe \xmlinline{<spintracking>}) übernommen.

%   Hier kann die Anzahl der Umläufe $N_u$ eingestellt werden, über die die Resonanzstärken
%   berechnet werden. Wenn der Standardwert 0 gewählt ist, wird die für die gewählte
%   Auflösung \xmlinline{<step>} benötigte Mindestanzahl $N_u = 1/\Delta\ga$ verwendet.
% \end{configdoc}


\clearpage
\appendix

\section{\ele Installation Guide}
\label{sec:ele-install}




\printbibliography[heading=bibintoc]

\end{document}



%%% Local Variables: 
%%% mode: Latex
%%% LaTeX-command: "latex -shell-escape"
%%% End: 
