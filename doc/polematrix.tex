%% polematrix documentation
%% Copyright (C) 2017 Jan Felix Schmidt <janschmidt@mailbox.org>
%%   
%% This program is free software: you can redistribute it and/or modify
%% it under the terms of the GNU General Public License as published by
%% the Free Software Foundation, either version 3 of the License, or
%% (at your option) any later version.
%%
%% This program is distributed in the hope that it will be useful,
%% but WITHOUT ANY WARRANTY; without even the implied warranty of
%% MERCHANTABILITY or FITNESS FOR A PARTICULAR PURPOSE.  See the
%% GNU General Public License for more details.
%%
%% You should have received a copy of the GNU General Public License
%% along with this program.  If not, see <http://www.gnu.org/licenses/>.

\documentclass[a4paper]{scrartcl}
\usepackage[english]{babel}
\usepackage[utf8]{inputenc}
\usepackage[margin=2.5cm]{geometry}
\usepackage{siunitx}
\usepackage{csquotes}
\usepackage{booktabs}
\usepackage[colorlinks]{hyperref}
\usepackage[backend=biber,style=alphabetic]{biblatex}

\usepackage{polem_defs}
\usepackage{cleveref}

\author{Jan Felix Schmidt \textless janschmidt@mailbox.org\textgreater}
\title{polematrix}
\subtitle{a spin tracking code for electron accelerators using elegant}
\date{documentation for version 0.99}

\hypersetup{
  pdftitle={polematrix documentation},
  pdfauthor={Jan Felix Schmidt}
}

\bibliography{polem}

\begin{document}
\maketitle

\begin{abstract}
  \polem is a spin tracking code. It calculates the motion of the spin of ultra
  relativistic electrons in particle accelerators. It implements a simple matrix tracking
  as multi-thread \cpp application. The input can be any lattice file from \ele or
  \madx. All required particle trajectories are also imported from the established
  particle tracking codes \ele or \madx.

  \polem includes synchrotron radiation via import from \ele or an own implementation of
  longitudinal phasespace. A short description of \polem is given in the IPAC'16
  contribution \cite{IPAC16-decoh} and in my phd thesis, which will be available soon (in
  German).
  
  Do not hesitate to contact me if you have any questions and please report bugs.
\end{abstract}

\tableofcontents
\clearpage



\section{Usage}
\label{sec:usage}

\polem is called via
\begin{bashcode}
  polematrix [options] [CONFIGURATION FILE]
\end{bashcode}
where \bashinline{[options]} are the command line options shown in
\cref{tab:polem-options}. All parameters of the spin tracking can be set up in an \xml
configuration file (\bashinline{[CONFIGURATION FILE]}) which is described in
\cref{sec:config}.


\begin{table}[h]
  \centering
  \begin{tabular}{ll}
    \toprule
    \bashinline{-h [ --help ]}              &  display this help message \\
    \bashinline{-V [ --version ]}           &  display version \\
    \bashinline{-T [ --template ]}          &  create config file template (\bashinline{template.pole}) and quit \\
    \bashinline{-R [ --resonance-strengths ]} &  estimate resonance strengths instead of spin tracking \\
    \midrule
    \bashinline{-t [ --threads ] arg (=all)}     &  set number of threads used for tracking \\
    \bashinline{-o [ --output-path ] arg (=.)}   &  path for output files \\
    \bashinline{-v [ --verbose ]}            &  activate additional status output \\
    \bashinline{-n [ --no-progressbar ]}     &  do not show progress bar during tracking\\
                                            &  (e.g. if output is redirected to a log file)\\
    \bashinline{-a [ --all ]}                &  write additional output files (e.g. lattice und orbit) \\
    \bashinline{-s [ --spintune ] arg}       &  in resonance strengths mode (\bashinline{-R}):\\
                                            &  calculate for given spin tune only\\
    \bottomrule
  \end{tabular}
  \caption{Command line options of \polem.}
  \label{tab:polem-options}
\end{table}



\section{Installation}
\label{sec:installation}

\polem has been tested only on Ubuntu/Debian Linux with the GCC compiler so far.

\subsection{Dependencies}
\label{sec:dependencies}

\polem requires the following programs and libraries:
\begin{itemize}
\item GCC compiler
\item CMake
\item Gnu Scientific Library (GSL) \cite{gsl}
\item Boost program options
\item Boost filesystem
\item Boost property tree \cite{boost-pt}
\item Boost random \cite{boost-random}
\item Armadillo library \cite{arma}
\item \pal library \cite{palattice}
\end{itemize}
\pal is available on github. The source code can be downloaded at \cite{palattice} and
compiled and installed as described in the enclosed README file.
%
All other dependencies may be in the package repositories of your Linux distribution. On
Ubuntu you can install all required packages with the following command:
\begin{bashcode}
  sudo apt-get install g++ cmake libgsl-dev libboost-dev libboost-program-options-dev libboost-filesystem-dev libarmadillo-dev
\end{bashcode}
Additionally, \polem relies on the established particle
tracking program \ele (or \madx), whose lattice files and tracking results are used for
spin tracking.  \ele (and \madx) can be configured and execute automatically by \polem using
\pal. The installation of \ele and \madx is described in \cref{sec:elemadx}.


    
\subsection{Build and Install \polem}
\label{sec:build}
\polem is compiled and installed using \software{CMake} with the following commands:
\begin{bashcode}
  cd polematrix/build
  cmake ../
  make
  sudo make install
\end{bashcode}
The default install path is \bashinline{/usr/local/bin}. It can be changed by setting the
\bashinline{CMAKE_INSTALL_PREFIX} variable in \bashinline{polematrix/CMakeLists.txt}.
%
To uninstall \polem run
\begin{bashcode}
  cd polematrix/build
  sudo make uninstall
\end{bashcode}



\subsection{Underlying Particle Tracking Programs}
\label{sec:elemadx}

I recommend to use \polem with \ele since much more tracking parameters can be set
automatically -- including number of particles (starting with a Gaussian profile in
radiation equilibrium), tracking duration and linear energy ramps. If you have only a
lattice file in \madx format, you can convert it automatically to an \ele lattice
using the program \software{convertlattice} which is included in \pal \cite{palattice}.

\subsubsection{\ele}
\label{sec:ele}
The particle tracking program \ele \cite{elegant} is developed at the Advanced Photon
Source at Argonne National Laboratory. Packages for many operating systems can be
downloaded at \cite{elegant-download}. Also install the \software{SDDSToolKit}, which is
needed to access the binary output files of \ele.
%
If there is no package for your system, you can download the \bashinline{Build-AOP-RPMs}
script instead which automatically builds all desired programs from source (see the guide
in \cref{sec:ele-install}).
%
There is a parallel version of \ele, called \software{Pelegant}, which speeds up the
tracking \cite{pelegant} (installation hints in \cref{sec:ele-install}). The \ele manual
can be found at \cite{elegant-manual}.

\subsubsection{\madx}
\label{sec:madx}
The particle tracking program \madx is developed at CERN and can be downloaded from the
website \cite{madx}, which also provides the \madx manual.




\section{Configuration File Reference}
\label{sec:config}


\clearpage
\appendix

\section{\ele Installation Guide}
\label{sec:ele-install}




\printbibliography[heading=bibintoc]

\end{document}



%%% Local Variables: 
%%% mode: Latex
%%% LaTeX-command: "latex -shell-escape"
%%% End: 
